\section{Opgave 3}
Rundetårn er 41.5 meter højt.

\subsection{Start med at lave et plot af rundetårns ydre fremtoning (idet der ses bort fra vinduer, døre, udsmykninger og den tilbyggede Trinitatiskirke). Du skal bruge parametriseringer af to omdrejningscylindre (hovedtårnet og observatoriets underdel), en cirkelskive (platformen) og en halvkugle (kuplen)}

Parameterfremstillingen for en cylinder er:
\begin{align}
    r(u,v) =(R \cdot cos(v), R \cdot sin(v), u) 
\end{align}
Hvor $u$ er en variabel som går fra 0 til højden af cylinderen og $v$ angiver gangen rundt i en cirkel og går fra 0 til $2\pi$. For bunden af rundetårn er højden 34,8 meter og for observatoriets bund er det 4 meter. 

Fladen som man står på er en flad ring som har parameterfremstillingen
\begin{align}
    r(u,v) = (u \cdot cos(v), u \cdot sin(v), 34.9)
\end{align}
Her er $u$ radius gående fra 3 til 7 og $v$ angiver gangen rundt i en cirkel og går fra 0 til $2\pi$. Tredje-koordinaten er 34,9 da det er der hvor den flade ring ligger i z-aksen.

Observatoriets halvkugle har parameterfremstillingen
\begin{align}
    r(u,v) = (sin(u)*3*cos(v),sin(u)*3*sin(v),cos(u)*3+38,4)
\end{align}
Hvor radius er 3 og centrum er blevet flyttet 38,4 op ad z-aksen.

Det endelige rundetårn er tegnet på figur \ref{rund}
\begin{figure}[htp]
    \centering
    \includegraphics[width=5cm]{Rundetårn.png}
        \caption{Rundetårn tegnet i Maple}
    \label{rund}
\end{figure}

\newpage

Rundetårn er kendt for sneglegangen.
\subsection{Først betragtes Sneglegangen som en regulær flade der opfylder at snitkurverne mellem fladen og en vilkårlig halvplan som udgår fra tårnets omdrejningsakse, er vandrette linjestykker. Bestem en parameterfremstilling for fladen og bestem dens areal}

\subsection{Antag at Peter den Store red på midten af sneglegangen. Hvor lang var turen ned? Kontrollér oplysningerne om den gennemsnitlige stigning yderst og inderst.}

\subsection{Antag at sneglegangens lodrette tykkelse aftager proportionalt med afstanden fra grundniveau, således at den nederst er 0.5 meter og øverst 0.4 meter. Giv en parameterfremstilling af sneglegangen opfattet som massivt rumligt område. Hvor stor en del udgør sneglegangens samlede masse af hele massen af rundetårn. (brug den i linket oplyste massefylde for brændt ler og sten.)?}